\section*{Diskuze}
Bohužel jsme zapomněli jsme změřit vzdálenost vzorku $^{241}$Am od terčíku a hodnota \SI{3.0(5)}{\cm} je odhad, který v rámci nejistoty považujeme za správný. Velká chyba $r$ se projevila velkou chybou absolutní aktivity $A$. Velikost terčíku $S$ jsme zjistili od spolužáků, nicméně je možné, že byl použit jiný terčík, a proto hodnotu bereme s rezervou a nevěříme jí.



Funkci $f(T)$ jsme určili až na škálovací faktor způsobený nepřesným $r$. Přesto závislost přibližně odpovídá teoretické závislosti \eqref{teoreticka}, což napovídá, že jsme vzdálenost $r$ odhadli správně.

V numerické derivaci \eqref{dTdP} odčítáme ve jmenovateli blízká čísla, což způsobuje chybu, pokud jsme tlak nezměřili přesně.

Tlak vzduchu byl $P_0=\SI{960}{\hecto\pascal}$, což bylo způsobeno jinou teplotou a vlhkostí než je normální ($P_0=\SI{1013}{\hecto\pascal}$). Proto jsme i specifické ionizační ztráty $f$ vztahovali ve vzorci \ref{f} k tlaku \SI{960}{\hecto\pascal} a dostali jsme funkci, jejíž teoretický tvar se může lišit od \eqref{teoreticka}.



Naměřené energie $\alpha$-částic vylétajících ze vzorku mají o trochu nižší energii než tabelovanou, což mohlo být způsobeno vdáleností od terčíku a nedokonalým vakuem.